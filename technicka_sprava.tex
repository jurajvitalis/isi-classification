% !TeX program = pdflatex
% !TeX encoding = utf-8
% !TeX spellchek = sk_SK

\documentclass[12pt]{article}

\usepackage[slovak]{babel}
\usepackage[utf8]{inputenc}
\usepackage[T1]{fontenc}


\begin{document}
    \thispagestyle{empty}

	\centerline{{\Huge Zadanie \v{c}. {1}}}

    \bigskip

    \centerline{{\Large Inteligentné systémy v informatike 2021}}

    \section{Klasifikátor}

    Klasifikátor SVC je založený na algoritme Support Vector Machines (SVM).

    Maximalizačný problém, ktorý SVM rieši spočíva v nájdení hranice rozdeľujúcej dáta, ktorá
    spĺńa podmienku maximálnej vzdialenosti (marginu) medzi podpornými vektormi (Support vectors).

    \section{Preprocessing}

    \subsection{Preprocessing ordinálnych dát}
        Dataset v zadaní obsahuje číselné dáta (float), v ktorých niektoré 
        údaje chýbajú.
        Na doplnenie chýbajúcich dát som použil metódu imputácie, v ktorej
        sa chýbajúce hodnoty nahrádzajú priemernou hodnotou daného stĺpca.
        Imputáciu som implementoval pomocou funkcie \emph{SimpleImputer()} z knižnice scikit-learn.

        Aby klasifikátor nemal problém s naučením sa správnej hranice, je potrebné 
        číselné dáta, ktoré mu poskytneme na naučenie škálovať. Na škálovanie som použil
        funkciu \emph{StandardScaler()} z knižnice scikit-learn.

    \subsection{Preprocessing nominálnych dát}
        Druhá čast datasetu pozostáva z nominálnych hodnôt, konkrétne reťazcov.
        Kedže klasifikátor na učenie vyžaduje na vstupe číselné hodnoty, je potrebné tieto
        reťazce transformovať. Na túto úlohu som použil funkciu
        \emph{OneHotEncoder()} (OHE) z knižnice scikit-learn. OHE pre každý unikátny
        reťazec vytvorí samostatný stĺpec, v ktorom pre každú vzorku sa nachádza 1 pre daný string, 0 ináč.

    \newpage
    \thispagestyle{empty}

    \section{Výsledky}

    \subsection{Voľba klasifikátora}
    Ako stratégiu na nájdenie vhodného klasifikátora pre daný dataset som zvolil metódu
    \emph{Gridsearch}, ktorá spočíva v brute-force skúšaní všetkých kombinácií zadaných parametrov
    pre zadaný klasifikátor.
    Túto metódu som implementoval pomocou slučky, v ktorej som volal funkciu \emph{GridSearchCV()} z knižnice
    scikit-learn v každej iterácii na iný klasifikátor s inou množinou parametrov.

    Klasifikátory, ktoré som touto metódou vyskúšal boli:
    \emph{DecisionTreeClassifier()},
    \emph{RandomForestClassifier()},
    \emph{LinearSVC()},
    \emph{NuSVC()},
    \emph{SVC()}.

    Víťazom v GridSearchCV() sa stal klasifikátor SVC(), ktorý 
    pre konkrétne parametre \emph{SVC(C=0.0005, coef0=0, kernel='poly', degree='2')}
    dosiahol presnosť 0.985.

    \subsection{Voľba parametrov pre klasifikátor}
    Po natrénovaní SVC modelu s parametrami z GridSearchCV() mi model predikoval dáta v 
    testovacej zložke s chabou priemernou presnosťou 0.5 pri použití metriky roc\_auc\_score
    na 100 rôznych rozdeleniach.

    Došiel som k záveru, že parametre ktoré som dostal z GridSearchCV silno korelovali s 
    test/train rozdelením datasetu, na ktorom bol GridSearchCV volaný a mohlo dôjsť k pretrénovaniu.

    Po zmene parametra \emph{C} model \emph{SVC(C=1, coef0=0, kernel='poly', degree=2)} pri metrike roc\_auc\_score 
    dosahoval na 100 rôznych rozdeleniach priemernú presnosť 0.976.

    Tento model som nakoniec zvolil aj na finálnu predikciu X\_eval dát.


    \vfill
    TUKE 2021
    \hfill
    Juraj Vitališ

\end{document}
